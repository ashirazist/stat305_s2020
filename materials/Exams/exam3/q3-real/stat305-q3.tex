% For LaTeX-Box: root = stat305-pexam1.Rnw
\documentclass[addpoints]{examsetup}

\usepackage{etoolbox}
\usepackage{tikz,pgfplots}

\makeatletter
\def\maxwidth{ %
  \ifdim\Gin@nat@width>\linewidth
    \linewidth
  \else
    \Gin@nat@width
  \fi
}
\makeatother
\usepackage{pdfpages} 
\definecolor{fgcolor}{rgb}{0.345, 0.345, 0.345}
\newcommand{\hlnum}[1]{\textcolor[rgb]{0.686,0.059,0.569}{#1}}%
\newcommand{\hlstr}[1]{\textcolor[rgb]{0.192,0.494,0.8}{#1}}%
\newcommand{\hlcom}[1]{\textcolor[rgb]{0.678,0.584,0.686}{\textit{#1}}}%
\newcommand{\hlopt}[1]{\textcolor[rgb]{0,0,0}{#1}}%
\newcommand{\hlstd}[1]{\textcolor[rgb]{0.345,0.345,0.345}{#1}}%
\newcommand{\hlkwa}[1]{\textcolor[rgb]{0.161,0.373,0.58}{\textbf{#1}}}%
\newcommand{\hlkwb}[1]{\textcolor[rgb]{0.69,0.353,0.396}{#1}}%
\newcommand{\hlkwc}[1]{\textcolor[rgb]{0.333,0.667,0.333}{#1}}%
\newcommand{\hlkwd}[1]{\textcolor[rgb]{0.737,0.353,0.396}{\textbf{#1}}}%
\let\hlipl\hlkwb

\usepackage{ulem}

\usepackage{framed}
\makeatletter
\newenvironment{kframe}{%
 \def\at@end@of@kframe{}%
 \ifinner\ifhmode%
  \def\at@end@of@kframe{\end{minipage}}%
  \begin{minipage}{\columnwidth}%
 \fi\fi%
 \def\FrameCommand##1{\hskip\@totalleftmargin \hskip-\fboxsep
 \colorbox{shadecolor}{##1}\hskip-\fboxsep
     % There is no \\@totalrightmargin, so:
     \hskip-\linewidth \hskip-\@totalleftmargin \hskip\columnwidth}%
 \MakeFramed {\advance\hsize-\width
   \@totalleftmargin\z@ \linewidth\hsize
   \@setminipage}}%
 {\par\unskip\endMakeFramed%
 \at@end@of@kframe}
\makeatother

\definecolor{shadecolor}{rgb}{.97, .97, .97}
\definecolor{messagecolor}{rgb}{0, 0, 0}
\definecolor{warningcolor}{rgb}{1, 0, 1}
\definecolor{errorcolor}{rgb}{1, 0, 0}
\newenvironment{knitrout}{}{} % an empty environment to be redefined in TeX

\usepackage{alltt}
\usepackage{graphicx, fancyhdr}
\usepackage{amsmath, amsfonts}
\usepackage{color}
\usepackage{hyperref}

\newcommand{\blue}[1]{{\color{blue} #1}}

\setlength{\topmargin}{-.5 in} 
\setlength{\textheight}{9 in}
\setlength{\textwidth}{6.5 in} 
\setlength{\evensidemargin}{0 in}
\setlength{\oddsidemargin}{0 in} 
\setlength{\parindent}{0 in}
\newcommand{\ben}{\begin{enumerate}}
\newcommand{\een}{\end{enumerate}}



%% For LaTeX-Box: root = stat105_exam1_info.tex 
%%%%%%%%%%%%%%%%%%%%%%%%%%%%%%%%%%%%%%%%%%%%%%%%%%%%%%%%%%%%%%%%%%%%%%%%%%%%%%%%
%  File Name: stat105_exam1_info.tex
%  Purpose:
%
%  Creation Date: 24-09-2015
%  Last Modified: Thu Sep 24 13:51:36 2015
%  Created By:
%%%%%%%%%%%%%%%%%%%%%%%%%%%%%%%%%%%%%%%%%%%%%%%%%%%%%%%%%%%%%%%%%%%%%%%%%%%%%%%%
\newcommand{\course}[1]{\ifstrempty{#1}{STAT 105}{STAT 105, Section #1}}
\newcommand{\sectionNumber}{B}
\newcommand{\examDate}{October 1, 2015}
\newcommand{\semester}{FALL 2015}
\newcommand{\examNumber}{II}

\newcommand{\examTitle}{Exam \examNumber}

\runningheader{\course{\sectionNumber}}{Exam \examNumber}{\examDate}
\runningfooter{}{}{Page \thepage of \numpages}

\newcommand{\examCoverPage}{
   \begin{coverpages}
   \centering
   {\bfseries\scshape\Huge Exam I \par}
   \vspace{1cm}
   {\bfseries\scshape\LARGE \course{\sectionNumber} \par}
   {\bfseries\scshape\LARGE \semester \par}

   \vspace{2cm}

   \fbox{\fbox{\parbox{5.5in}{\centering 

      \vspace{.25cm} 
      
      {\bfseries\Large Instructions} \\

      \vspace{.5cm} 

      \begin{itemize}
         \item  The exam is scheduled for 80 minutes, from 8:00 to 9:20 AM. At 9:20 AM the exam will end.\\
         \item  A forumula sheet is attached to the end of the exam. Feel free to tear it off.\\
         \item  You may use a calculator during this exam.\\
         \item  Answer the questions in the space provided. If you run out of room, continue on the back of the page. \\
         \item  If you have any questions about, or need clarification on the meaning of an item on this exam, please ask your instructor. No other form of external help is permitted attempting to receive help or provide help to others will be considered cheating.\\
         \item  {\bfseries Do not cheat on this exam.} Academic integrity demands an honest and fair testing environment. Cheating will not be tolerated and will result in an immediate score of 0 on the exam and an incident report will be submitted to the dean's office.\\
      \end{itemize}

   }}}

   \vspace{2cm}

   \makebox[0.6\textwidth]{Name:\enspace\hrulefill}

   \vspace{1cm}

   \makebox[0.6\textwidth]{Student ID:\enspace\hrulefill}
   \end{coverpages}

}


\newcommand{\course}[1]{\ifstrempty{#1}{STAT 305}{STAT 305, Section #1}}
\newcommand{\sectionNumber}{7}
\newcommand{\examDate}{April 2, 2020}
\newcommand{\semester}{Spring 2020}
\newcommand{\examNumber}{III}

\usepackage{Sweave}
\begin{document}
\Sconcordance{concordance:stat305-q3.tex:stat305-q3.Rnw:%
1 81 1 1 0 3 1 1 9 139 1}


%-- : R code (Code in Document)


\examCoverPage

\begin{questions}
\question
Suppose that an eddy current nondestructive evaluation technique for identifying cracks in critical metal parts has a probability of about $.20$ of detecting a single crack of length $.003$in. in a certain material. Let $Y$ be the number of specimens inspected in order to obtain the first crack detection.
\vspace{1cm}
         \begin{subparts}
                  \subpart[2] Precisely state the distribution of $Y$, giving the values of any parameters necessary.
                  \vspace{2cm}
                  
                  \subpart[3] Find the probability that $P(Y = 5)$
                  \hfill \fbox{$P(Y = 5)=$ \hspace{3cm}}
                  \vspace{5cm}
                  
                  \subpart[3] Find the probability that $P(Y \le 4)$
                  \hfill \fbox{$P(Y \le 4)=$ \hspace{3cm}}
                  \vspace{5cm}
                  
                  \subpart[2] What is $\text{E}Y$?
                  \hfill \fbox{$\text{E}(Y)= $ \hspace{3cm}}
                  \vspace{2cm}
                  
                           
         \end{subparts}
\newpage
\question 

Suppose that $X$ is a continuous random variable with cumulative density function (\emph{cdf}):
$$
F(x) = 
\begin{cases}
    0 &  x < 0 \\
   1 - e^{-3x} &  x \ge 0
\end{cases}
$$

            \begin{parts}
            
            \part[3] What is the probability that $X$ takes a value greater than 2?
            \hfill \fbox{ \textcolor[rgb]{1.00,1.00,1.00}{$\bigcap$} \hskip-0.4cm $P=$ \hspace{2cm}}

                 \vspace{4cm}
            
            \part[3] Derive f(x), the probability density function.
            
                 \vspace{5cm}
            \part[4] Find the expected value of X.
            \hfill \fbox{ \textcolor[rgb]{1.00,1.00,1.00}{$\bigcap$} \hskip-0.4cm $\text{E}(X)=$ \hspace{2cm}}
            
                 \vspace{5cm}            
            \end{parts}
\newpage
\question
 Suppose that $X_1, X_2, X_3, X_4, X_5$ and $Y_1, Y_2, Y_3, Y_4, Y_5$ are all independent random variables where for any $i$
   $$E(X_i) = \mu_X$$
   $$Var(X_i) = \sigma_X^2$$
   $$E(Y_i) = \mu_Y$$
   $$Var(Y_i) = \sigma_Y^2$$
   Suppose that we define a random variable $U$ to help compare the values taken by $X_i$s and the values taken by the $Y_i$s by pairing the random variables like this:
   $$U = \frac{1}{5}(X_1 - Y_1) + \frac{1}{5}(X_2 - Y_2) + \frac{1}{5}(X_3 - Y_3) + \frac{1}{5}(X_4 - Y_4) + \frac{1}{5}(X_5 - Y_5) $$
       \begin{parts}
           \part[3] Find the mean of $U$ (hint: it will include $\mu_X$ and $\mu_Y$.) \vspace{5cm}
            \hfill \fbox{ \textcolor[rgb]{1.00,1.00,1.00}{$\bigcap$} \hskip-0.4cm $\text{E}(U)=$ \hspace{2cm}}
           
           \vspace{4cm}
        
           \part[4] Find the standard deviation of $U$ (hint: it will include $\sigma_X^2$ and $\sigma_Y^2$)
            \hfill \fbox{ \textcolor[rgb]{1.00,1.00,1.00}{$\bigcap$} \hskip-0.4cm $\text{SD}(U)=$ \hspace{2cm}}
           
           \vspace{4cm}
       \end{parts}
\newpage


\question
Consider the following joint distribution for two random variables X and Y:

 \begin{table}[h!]
     \centering
     \begin{tabular}{lllll}
        \hline
         Y \textbackslash  X  &  0    &  1   & 2 & 3 \\\hline \hline
         0      &  0.1 &  0.2 & 0.1 & 0   \\
         1      &  0.4  & 0 & 0 & 0.2   \\ \hline\hline

     \end{tabular}
  \end{table}
   \begin{parts}
   
      \part[4] Find the marginal pmfs of X and Y

            \begin{tabular}{c|c m{1cm} c m{1cm} c c c c c c c}
               
               x  & & &  & & &  & & & &\\ 
               \hline
               f(x)  & & &  & & &  & & & &\\
            \end{tabular}
           \quad   \quad   \quad \quad   \quad   \quad \quad   \quad   \quad
            \begin{tabular}{c|c m{1cm} c m{1cm} c c c c c c c}
               
               y  & & &  & & &  & & & &\\ 
               \hline
               f(y)  & & &  & & &  & & & &\\
            \end{tabular}
      \vspace{4cm}
      
      \part[4] Find the conditional distribution of $f_{X\vert Y}(x\vert y=1)$
      \begin{table}[h]
            \centering
            \begin{tabular}{c|c m{5cm} c  c  c c c c c c}
               
               x  & & &  & & &  & & & &\\ 
               \hline
               f_{X\vert Y}(x\vert y=1)  & & &  & & &  & & & &\\
            \end{tabular}
      \end{table}
      \vspace{4cm}
      
      \part[3] Find the conditional expected value of $x\vert y=1$. That is $\text{E}_{(X\vert Y)}(x\vert y=1)$\\
      \hfill \fbox{ \textcolor[rgb]{1.00,1.00,1.00}{$\bigcap$} \hskip-0.4cm $\text{E}_{(X\vert Y)}(x\vert y=1)=$ \hspace{4cm}}
      \vspace{3cm}
      
      \part[4] Find the conditional variance  of $x\vert y=1$. That is $\text{Var}_{(X\vert Y)}(x\vert y=1)$\\
      \hfill \fbox{ \textcolor[rgb]{1.00,1.00,1.00}{$\bigcap$} \hskip-0.4cm $\text{Var}_{(X\vert Y)}(x\vert y=1)=$ \hspace{4cm}}
      \vspace{6cm}
       \part[3] Are X and Y independent? Why or why not?
       \vspace{4cm}
   \end{parts}

\question
Let $X$ be a normal random variable with a mean of -2 and a varaince of 16 (i.e., $X \sim N(-2, 16)$) Find the following probabilities (note: the attached standard normal probability table may be helpful):

          \begin{parts}
          \part[3] $P(-8 \le X < 1)$
          \hfill \fbox{ \textcolor[rgb]{1.00,1.00,1.00}{$\bigcap$} \hskip-0.4cm $\text{P}=$ \hspace{2cm}}
          \vspace{4cm}
          \part[4] $P(|X + 2| \ge 4)$ 
          \hfill \fbox{ \textcolor[rgb]{1.00,1.00,1.00}{$\bigcap$} \hskip-0.4cm $\text{P}=$ \hspace{2cm}}
          \vspace{5cm}
          \part[4] Find the value of c such that $P(|X + 1| \le c) = 0.95$
          \hfill \fbox{ \textcolor[rgb]{1.00,1.00,1.00}{$\bigcap$} \hskip-0.4cm $\text{c}=$ \hspace{2cm}}
          \end{parts}
\newpage
\question
Seventy independent messages are sent from an electronic transmission center.  Messages are processed sequentially, one after another.  Transmission time of each message is Exponential with parameter $\alpha = 5$min.  
\begin{parts}
\part[3] what are the expected value and variance of the sample mean of all 70 messages? 

\hfill \fbox{ \textcolor[rgb]{1.00,1.00,1.00}{$\bigcap$} \hskip-0.4cm $\text{E}(\overline{X}) =$ \hspace{2.35cm}}

\hfill \fbox{ \textcolor[rgb]{1.00,1.00,1.00}{$\bigcap$} \hskip-0.4cm $\text{Var}(\overline{X}) =$ \hspace{2cm}}
\vspace{4cm}

\part[3] Find the probability that the average of all 70 messages are transmitted in less than 12 minutes. 
\hfill \fbox{ \textcolor[rgb]{1.00,1.00,1.00}{$\bigcap$} \hskip-0.4cm $\text{P} =$ \hspace{2cm}}
\vspace{4cm}


   \end{questions}
\end{document}
