\documentclass[10pt,landscape]{article}
\usepackage{multicol}
\usepackage{amsmath}
\usepackage{calc}
\usepackage{ifthen}
\usepackage[landscape]{geometry}
\usepackage{hyperref}

% To make this come out properly in landscape mode, do one of the following
% 1.
%  pdflatex latexsheet.tex
%
% 2.
%  latex latexsheet.tex
%  dvips -P pdf  -t landscape latexsheet.dvi
%  ps2pdf latexsheet.ps

% This sets page margins to .5 inch if using letter paper, and to 1cm
% if using A4 paper. (This probably isn't strictly necessary.)
% If using another size paper, use default 1cm margins.
\ifthenelse{\lengthtest { \paperwidth = 11in}}
	{ \geometry{top=.5in,left=.5in,right=.5in,bottom=.5in} }
	{\ifthenelse{ \lengthtest{ \paperwidth = 297mm}}
		{\geometry{top=1cm,left=1cm,right=1cm,bottom=1cm} }
		{\geometry{top=1cm,left=1cm,right=1cm,bottom=1cm} }
	}

% Turn off header and footer
\pagestyle{empty}
 

% Redefine section commands to use less space
\makeatletter
\renewcommand{\section}{\@startsection{section}{1}{0mm}%
                                {2ex} %{-1ex plus -.5ex minus -.2ex}%
                                {0.5ex plus .2ex}%x
                                {\normalfont\large\bfseries\underline}}
\renewcommand{\subsection}{\@startsection{subsection}{2}{0mm}%
                                {1ex} %{-1ex plus -.5ex minus -.2ex}%
                                {0.8ex} %{0.5ex plus .2ex}%
                                {\normalfont\normalsize\bfseries}}
\renewcommand{\subsubsection}{\@startsection{subsubsection}{3}{0mm}%
                                {-1ex plus -.5ex minus -.2ex}%
                                {1ex plus .2ex}%
                                {\normalfont\small\bfseries}}
\makeatother

% Define BibTeX command
\def\BibTeX{{\rm B\kern-.05em{\sc i\kern-.025em b}\kern-.08em
    T\kern-.1667em\lower.7ex\hbox{E}\kern-.125emX}}

% Don't print section numbers
\setcounter{secnumdepth}{0}


\setlength{\parindent}{0pt}
\setlength{\parskip}{0pt plus 0.5ex}

% -----------------------------------------------------------------------

\begin{document}

\raggedright
\footnotesize
\begin{multicols}{2}


% multicol parameters
% These lengths are set only within the two main columns
%\setlength{\columnseprule}{0.25pt}
\setlength{\premulticols}{1pt}
\setlength{\postmulticols}{1pt}
\setlength{\multicolsep}{1pt}
\setlength{\columnsep}{2pt}

\begin{center}
   \Large{\textbf{STAT 305 Quiz II}} \\
   \Large{\textbf{ Reference Sheet  }}\\
\end{center}

\section{Numeric Summaries}
\begin{tabular}{@{}ll@{}}
        & \\
mean    & $\bar{x} = \frac{1}{n}\sum_{i=1}^n x_i$ \\
        & \\
population variance  & $\sigma^2 = \frac{1}{n}\sum_{i=1}^n \left(x_i - \bar{x} \right)^2$ \\
        & \\
population standard deviation  & $\sigma = \sqrt{\frac{1}{n}\sum_{i=1}^n \left(x_i - \bar{x} \right)^2}$ \\
        & \\
sample variance  & $s^2 = \frac{1}{n-1}\sum_{i=1}^n \left(x_i - \bar{x} \right)^2$ \\
        & \\
sample standard deviation  & $s = \sqrt{\frac{1}{n-1}\sum_{i=1}^n \left(x_i - \bar{x} \right)^2}$ \\
        & \\
\end{tabular}

\vspace{2cm}

\section{Functions}
\vspace{0.3cm}

\textbf{Quantile Function $Q(p)$} For a univariate sample consisting of
\(n\) values that are ordered so that \(x_1 \le x_2 \le \ldots \le x_n\)
and value \(p\) where \(0 \le p \le 1\), let
\(i = \lfloor n \cdot p + 0.5 \rfloor\). Then the quantile function at
\(p\) is:

\[
      Q(p) = 
      \begin{cases}
     \small x_i & \small \lfloor n \cdot p + .5 \rfloor = n \cdot p + .5 \\
      \small x_i  +\left( n p - i + .5 \right) \left( x_{i+1} - x_i \right) &\small  \lfloor n \cdot p + .5 \rfloor \ne n \cdot p + .5 \\
      \end{cases}
\]

\textbf{Measures of Central Tendency}

\begin{itemize}
\item
  \(\small Q\left(\frac{1-.5}{n}\right)\) is called the \textbf{minimum}
  and \(\small Q\left(\frac{n-.5}{n}\right)\) is called the
  \textbf{maximum} of a distribution.
\item
  \(\small Q(.5)\) is called the \textbf{median} of a distribution.
\item
  \(\small Q(.25)\) and \(\small Q(.75)\) are called the \textbf{first
  (or lower) quartile} and \textbf{third (or upper) quartile} of a
  distribution, respectively.
\item
  The \textbf{interquartile range (IQR)} is defined as
  \[\small IQR = Q(.75) - Q(.25)\].
\item
  An \textbf{outlier} is a data point that is larger than
  \(\small Q(.75) + 1.5*IQR\) or smaller than
  \(\small Q(.25) - 1.5*IQR\).
\end{itemize}

\end{multicols}
\end{document}

