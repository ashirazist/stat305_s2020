% For LaTeX-Box: root = stat305-pexam1.Rnw
\documentclass[addpoints]{examsetup}

\usepackage{etoolbox}
\usepackage{tikz,pgfplots}

\makeatletter
\def\maxwidth{ %
  \ifdim\Gin@nat@width>\linewidth
    \linewidth
  \else
    \Gin@nat@width
  \fi
}
\makeatother
\usepackage{pdfpages} 
\definecolor{fgcolor}{rgb}{0.345, 0.345, 0.345}
\newcommand{\hlnum}[1]{\textcolor[rgb]{0.686,0.059,0.569}{#1}}%
\newcommand{\hlstr}[1]{\textcolor[rgb]{0.192,0.494,0.8}{#1}}%
\newcommand{\hlcom}[1]{\textcolor[rgb]{0.678,0.584,0.686}{\textit{#1}}}%
\newcommand{\hlopt}[1]{\textcolor[rgb]{0,0,0}{#1}}%
\newcommand{\hlstd}[1]{\textcolor[rgb]{0.345,0.345,0.345}{#1}}%
\newcommand{\hlkwa}[1]{\textcolor[rgb]{0.161,0.373,0.58}{\textbf{#1}}}%
\newcommand{\hlkwb}[1]{\textcolor[rgb]{0.69,0.353,0.396}{#1}}%
\newcommand{\hlkwc}[1]{\textcolor[rgb]{0.333,0.667,0.333}{#1}}%
\newcommand{\hlkwd}[1]{\textcolor[rgb]{0.737,0.353,0.396}{\textbf{#1}}}%
\let\hlipl\hlkwb

\usepackage{ulem}

\usepackage{framed}
\makeatletter
\newenvironment{kframe}{%
 \def\at@end@of@kframe{}%
 \ifinner\ifhmode%
  \def\at@end@of@kframe{\end{minipage}}%
  \begin{minipage}{\columnwidth}%
 \fi\fi%
 \def\FrameCommand##1{\hskip\@totalleftmargin \hskip-\fboxsep
 \colorbox{shadecolor}{##1}\hskip-\fboxsep
     % There is no \\@totalrightmargin, so:
     \hskip-\linewidth \hskip-\@totalleftmargin \hskip\columnwidth}%
 \MakeFramed {\advance\hsize-\width
   \@totalleftmargin\z@ \linewidth\hsize
   \@setminipage}}%
 {\par\unskip\endMakeFramed%
 \at@end@of@kframe}
\makeatother

\definecolor{shadecolor}{rgb}{.97, .97, .97}
\definecolor{messagecolor}{rgb}{0, 0, 0}
\definecolor{warningcolor}{rgb}{1, 0, 1}
\definecolor{errorcolor}{rgb}{1, 0, 0}
\newenvironment{knitrout}{}{} % an empty environment to be redefined in TeX

\usepackage{alltt}
\usepackage{graphicx, fancyhdr}
\usepackage{amsmath, amsfonts}
\usepackage{color}
\usepackage{hyperref}

\newcommand{\blue}[1]{{\color{blue} #1}}

\setlength{\topmargin}{-.5 in} 
\setlength{\textheight}{9 in}
\setlength{\textwidth}{6.5 in} 
\setlength{\evensidemargin}{0 in}
\setlength{\oddsidemargin}{0 in} 
\setlength{\parindent}{0 in}
\newcommand{\ben}{\begin{enumerate}}
\newcommand{\een}{\end{enumerate}}



%% For LaTeX-Box: root = stat105_exam1_info.tex 
%%%%%%%%%%%%%%%%%%%%%%%%%%%%%%%%%%%%%%%%%%%%%%%%%%%%%%%%%%%%%%%%%%%%%%%%%%%%%%%%
%  File Name: stat105_exam1_info.tex
%  Purpose:
%
%  Creation Date: 24-09-2015
%  Last Modified: Thu Sep 24 13:51:36 2015
%  Created By:
%%%%%%%%%%%%%%%%%%%%%%%%%%%%%%%%%%%%%%%%%%%%%%%%%%%%%%%%%%%%%%%%%%%%%%%%%%%%%%%%
\newcommand{\course}[1]{\ifstrempty{#1}{STAT 105}{STAT 105, Section #1}}
\newcommand{\sectionNumber}{B}
\newcommand{\examDate}{October 1, 2015}
\newcommand{\semester}{FALL 2015}
\newcommand{\examNumber}{II}

\newcommand{\examTitle}{Exam \examNumber}

\runningheader{\course{\sectionNumber}}{Exam \examNumber}{\examDate}
\runningfooter{}{}{Page \thepage of \numpages}

\newcommand{\examCoverPage}{
   \begin{coverpages}
   \centering
   {\bfseries\scshape\Huge Exam I \par}
   \vspace{1cm}
   {\bfseries\scshape\LARGE \course{\sectionNumber} \par}
   {\bfseries\scshape\LARGE \semester \par}

   \vspace{2cm}

   \fbox{\fbox{\parbox{5.5in}{\centering 

      \vspace{.25cm} 
      
      {\bfseries\Large Instructions} \\

      \vspace{.5cm} 

      \begin{itemize}
         \item  The exam is scheduled for 80 minutes, from 8:00 to 9:20 AM. At 9:20 AM the exam will end.\\
         \item  A forumula sheet is attached to the end of the exam. Feel free to tear it off.\\
         \item  You may use a calculator during this exam.\\
         \item  Answer the questions in the space provided. If you run out of room, continue on the back of the page. \\
         \item  If you have any questions about, or need clarification on the meaning of an item on this exam, please ask your instructor. No other form of external help is permitted attempting to receive help or provide help to others will be considered cheating.\\
         \item  {\bfseries Do not cheat on this exam.} Academic integrity demands an honest and fair testing environment. Cheating will not be tolerated and will result in an immediate score of 0 on the exam and an incident report will be submitted to the dean's office.\\
      \end{itemize}

   }}}

   \vspace{2cm}

   \makebox[0.6\textwidth]{Name:\enspace\hrulefill}

   \vspace{1cm}

   \makebox[0.6\textwidth]{Student ID:\enspace\hrulefill}
   \end{coverpages}

}


\newcommand{\course}[1]{\ifstrempty{#1}{STAT 305}{STAT 305, Section #1}}
\newcommand{\sectionNumber}{3}
\newcommand{\examDate}{December , 2018}
\newcommand{\semester}{FALL 2019}
\newcommand{\examNumber}{IV}

\usepackage{Sweave}
\begin{document}
\Sconcordance{concordance:stat305-pq4.tex:stat305-pq4.Rnw:%
1 81 1 1 0 3 1 1 9 57 1 1 6 20 1}


%-- : R code (Code in Document)


\examCoverPage

\begin{questions}
\question
In an effort to understand smartphone use, the American Association of Psychologists gained access to a single day's data on 625 smartphone users. They found that the smartphone users sampled spent an average of 270 minutes using their phone on that day with a standard deviation of phone use was 32.15 minutes.

The following table may be useful:

\begin{table}[h]
   \centering
   \caption{z's for use in Two-sided Large-$n$ intervals for the mean}
   \begin{tabular}{cc}
      \hline \\
      Desired Confidence & $z$ \\
      \hline 
      80\% & 1.28 \\
      90\% & 1.645 \\
      95\% & 1.96 \\
      98\% & 2.33 \\
      99\% & 2.58 \\
      \hline 
   \end{tabular}
\end{table}
\begin{parts}
\part[4] Provide a two-sided 95\% confidence interval for true number of minutes smartphone users spend using thier phones on average. \vspace{4cm}
\part[4] Provide a one-sided lower bound 99\% confidence lower bound for the number of minutes smartphone users spend using thier phones on average. \vspace{4cm}
\part[10] A similar study two years ago determined that smartphone users spend an average of 4 hours per day using their phones. Perform a hypothesis test at the $\alpha = 0.05$ significance level for the claim that this is no longer the case.\\
\emph{Note: Write down all six steps for full credit.}

\end{parts}

\newpage\null\thispagestyle{empty}\newpage
\newpage

 

\question A group of scientists are trying to understand the effects of temperature on two O-ring designs for a rocket. By placing the O-ring (attached to a valve) in a chamber and slowly lowering the chamber's temperature, the scientists are able to record the temperature at which the O-ring fails by monitoring when the valve begins to leak.
After testing 10 O-rings for each type, the scientists find the mean failure temperature for the first O-ring design sample to be $50$ K with a sample variance of 10 and the mean failure temperature of the second O-ring sample to be $53$ K with a sample variance of $20$.
\begin{parts}
\part[4] Provide a one-sided lower bound 95\% confidence interval for the true failure temperature of the \textbf{first} O-ring design. \vspace{3cm}
\part[4] Provide a two-sided 95\% confidence interval for the true failure temperature of the \textbf{second} O-ring design. \vspace{3cm}
\part[6] Assume that the failure temperatures for both O-ring designs is normally distributed. Provide a one-sided upper bound 95\% confidence interval for \textbf{difference in the two designs}. \vspace{4cm}

\part[10] Conduct a hypothesis test at the $\alpha = 0.05$ significance level for the claim that the true difference between the population means of the two O-rings ($\mu_X- \mu_Y$)  are significantly different.\\
\emph{Note: Write down all six steps for full credit.}
\end{parts}
\newpage\null\thispagestyle{empty}\newpage

\newpage

\question
An arctic research station recently did a major overhaul to their server system hardware and the technicians are checking to make sure that there has been no loss in download speed.
The previous download speed had an average of 63.4 Mbps.
A systems analyst took 10 readings on the download speeds during the course of a day to check. 
Her results are below (in Mbps):


\begin{center}
62.87, 63.04, 62.83, 63.32, 63.07, 62.84, 63.1, 63.15, 63.12, 62.94
\end{center}


The sample average is 63.03 and the sample variance is 0.025.

\begin{parts}
\part[4] Provide a two-sided 90\% confidence interval for the mean download speed. \vspace{3cm}
\part[4] Provide a one sided 95\% lower confidence bound for the mean download speed. \vspace{3cm}
\part[10] Conduct a hypothesis test at the 95\% confidence level for the null hypothesis $\mu \ge 63.4$ against the alternative $\mu < 63.4$. Include your hypothesis statement, the choice of test statistic, the p-value, and your conclusion.\\
\emph{Note: Write down all six steps for full credit.}

\end{parts}


\newpage\null\thispagestyle{empty}\newpage

\end{questions}
\end{document}
