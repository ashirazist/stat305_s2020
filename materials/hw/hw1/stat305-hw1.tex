\documentclass[11pt]{article}\usepackage[]{graphicx}\usepackage[]{color}
%% maxwidth is the original width if it is less than linewidth
%% otherwise use linewidth (to make sure the graphics do not exceed the margin)
\makeatletter
\def\maxwidth{ %
  \ifdim\Gin@nat@width>\linewidth
    \linewidth
  \else
    \Gin@nat@width
  \fi
}
\makeatother

\definecolor{fgcolor}{rgb}{0.345, 0.345, 0.345}
\newcommand{\hlnum}[1]{\textcolor[rgb]{0.686,0.059,0.569}{#1}}%
\newcommand{\hlstr}[1]{\textcolor[rgb]{0.192,0.494,0.8}{#1}}%
\newcommand{\hlcom}[1]{\textcolor[rgb]{0.678,0.584,0.686}{\textit{#1}}}%
\newcommand{\hlopt}[1]{\textcolor[rgb]{0,0,0}{#1}}%
\newcommand{\hlstd}[1]{\textcolor[rgb]{0.345,0.345,0.345}{#1}}%
\newcommand{\hlkwa}[1]{\textcolor[rgb]{0.161,0.373,0.58}{\textbf{#1}}}%
\newcommand{\hlkwb}[1]{\textcolor[rgb]{0.69,0.353,0.396}{#1}}%
\newcommand{\hlkwc}[1]{\textcolor[rgb]{0.333,0.667,0.333}{#1}}%
\newcommand{\hlkwd}[1]{\textcolor[rgb]{0.737,0.353,0.396}{\textbf{#1}}}%
\let\hlipl\hlkwb

\usepackage{ulem}

\usepackage{framed}
\makeatletter
\newenvironment{kframe}{%
 \def\at@end@of@kframe{}%
 \ifinner\ifhmode%
  \def\at@end@of@kframe{\end{minipage}}%
  \begin{minipage}{\columnwidth}%
 \fi\fi%
 \def\FrameCommand##1{\hskip\@totalleftmargin \hskip-\fboxsep
 \colorbox{shadecolor}{##1}\hskip-\fboxsep
     % There is no \\@totalrightmargin, so:
     \hskip-\linewidth \hskip-\@totalleftmargin \hskip\columnwidth}%
 \MakeFramed {\advance\hsize-\width
   \@totalleftmargin\z@ \linewidth\hsize
   \@setminipage}}%
 {\par\unskip\endMakeFramed%
 \at@end@of@kframe}
\makeatother

\definecolor{shadecolor}{rgb}{.97, .97, .97}
\definecolor{messagecolor}{rgb}{0, 0, 0}
\definecolor{warningcolor}{rgb}{1, 0, 1}
\definecolor{errorcolor}{rgb}{1, 0, 0}
\newenvironment{knitrout}{}{} % an empty environment to be redefined in TeX

\usepackage{alltt}
\usepackage{graphicx, fancyhdr}
\usepackage{amsmath, amsfonts}
\usepackage{color}
\usepackage{hyperref}

\newcommand{\blue}[1]{{\color{blue} #1}}

\setlength{\topmargin}{-.5 in} 
\setlength{\textheight}{9 in}
\setlength{\textwidth}{6.5 in} 
\setlength{\evensidemargin}{0 in}
\setlength{\oddsidemargin}{0 in} 
\setlength{\parindent}{0 in}
\newcommand{\ben}{\begin{enumerate}}
\newcommand{\een}{\end{enumerate}}


\lhead{STAT 305}
\chead{Homework 1} 
\rhead{Due Thursday, September 5th in the class}
\lfoot{Fall 2019} 
\cfoot{\thepage} 
\rfoot{} 
\renewcommand{\headrulewidth}{0.4pt} 
\renewcommand{\footrulewidth}{0.4pt} 

\def\Exp#1#2{\ensuremath{#1\times 10^{#2}}}
\def\Case#1#2#3#4{\left\{ \begin{tabular}{cc} #1 & #2 \phantom
{\Big|} \\ #3 & #4 \phantom{\Big|} \end{tabular} \right.}
\IfFileExists{upquote.sty}{\usepackage{upquote}}{}
\begin{document}
\pagestyle{fancy} 

Show \textbf{all} of your work on this assignment and answer each question fully in the given context. 
Each individual part is worth 5 points and partial credit is awarded for close answers. 
Regardless of the total number of points on this homework, it will have the same weight as all other homeworks in terms of its impact on course grade. 
So, if a specific homework has 50 points and you lose 5, then your grade on that homework will be 45/50=90\%.
If there are 10 points on a homework and you lose 5, then your grade on that homework will be 5/10=50\%.
The average of those two homeworks would be $(90 + 50)/2 = 70$.

\emph{Please} staple your assignment!

\begin{itemize}

\item \textbf{Problem1: Chapter 1, Section 2, Exercise 2 (page 13)} \\

\item \textbf{Problem2: Chapter 1, Section 2, Exercise 3 (page 13)}\\

\item \textbf{Problem3: Chapter 1, End of chapter execise, Exercise 8 (page 24)}\\

\item \textbf{Problem4: Chapter 2, Section 2, Exercise 3 (page 37)}\\


\item \textbf{Problem5: Hockey game attendance.}

Caroline performs the following study to see if outside temperature
has an effect on attendance at her college's hockey games. For each
hockey game at her college, Caroline records the outside temperature and the attendance. Here are her results:

\hspace{1in}
\begin{tabular}{|cr@{/}r|c|c|} \hline
\multicolumn{3}{|c|}{\emph{Day of Week}} & \emph{Temperature, deg. F} & \emph{Attendance} \\ \hline
Friday & 12&14 & 35 & 840 \\
Wednesday & 12&19 & 20 & 560 \\
Tuesday & 1&8 & $\!-5$ & 340 \\
Friday & 1&11 & 23 & 775 \\
Wednesday & 1&23 & 14 & 680 \\
Saturday & 2&2 & 30 & 950 \\
Friday & 2&8 & 28 & 950 \\
\hline
\end{tabular}
 
\ben 
\item Is this an experiment or observational study?

\item What type of variable is attendance?

\een

Caroline analyzes her results and finds that
outside temperature and attendance have a strong
positive correlation (i.e., as one increases, the other also increases).
She concludes that higher game day temperatures causes higher
attendance at their college's hockey games.

\ben
\addtocounter{enumii}{2}
\item Did she come to a proper conclusion for this study? Why or why not?

\item Look at the day of the week of the hockey games.
What type of variable is this?
 
\item 
Rewrite the data table, adding a new column ``School Night" (using the values ``no" if the game is on a Friday or Saturday, and ``yes" if the game is on any other day).
How does Attendance relate to School Night?

\een

\item \textbf{Problem6: Washer stretching.}
  
George works for a company that manufactures 
rubber washers.  He randomly selects 1000 
washers off the assembly line throughout two weeks
for a study on the durability of these washers under  
stretching.  To make sure that the washers 
are fit to be used in the real world, George must test 
the washers.  Holding heat constant, 
George subjects each washer to one of various methods of 
stretching.  The washers are randomly assigned to be stretched under one of five different forces (low, medium-low, medium, medium-high, and high).
After each test, George classifies 
a washer as either defective or non-defective. 
 
\ben 
\item Is this an experiment or observational study? 

\item What type of variable is heat? 
 
\item What type of variable is the amount of stretching? 
 
\item What type of variable is response to the stretching method?
 
\item
The 1000 selected washers constitutes the sample. What is the population? 

\item George analyzes the results and finds that the defect rate
increases with the amount of stretching.
Can George conclude that the amount of stretching 
causes a change in the defect rate of the washers?
Why or why not?

%Remember, though, that there always exists the chance that the sample
%was not properly representative of the population, and perhaps 
%the 100 washers he selected 
%were some of the few washers in the population
%that showed no effect. 
%But through proper
%experimental design, this chance is minimized and George can 
%be highly confident in the results. More on confidence in Chapter 6.
\een


\end{itemize}


\end{document}
