\documentclass[11pt]{article}\usepackage[]{graphicx}\usepackage[]{color}
%% maxwidth is the original width if it is less than linewidth
%% otherwise use linewidth (to make sure the graphics do not exceed the margin)
\makeatletter
\def\maxwidth{ %
  \ifdim\Gin@nat@width>\linewidth
    \linewidth
  \else
    \Gin@nat@width
  \fi
}
\makeatother

\definecolor{fgcolor}{rgb}{0.345, 0.345, 0.345}
\newcommand{\hlnum}[1]{\textcolor[rgb]{0.686,0.059,0.569}{#1}}%
\newcommand{\hlstr}[1]{\textcolor[rgb]{0.192,0.494,0.8}{#1}}%
\newcommand{\hlcom}[1]{\textcolor[rgb]{0.678,0.584,0.686}{\textit{#1}}}%
\newcommand{\hlopt}[1]{\textcolor[rgb]{0,0,0}{#1}}%
\newcommand{\hlstd}[1]{\textcolor[rgb]{0.345,0.345,0.345}{#1}}%
\newcommand{\hlkwa}[1]{\textcolor[rgb]{0.161,0.373,0.58}{\textbf{#1}}}%
\newcommand{\hlkwb}[1]{\textcolor[rgb]{0.69,0.353,0.396}{#1}}%
\newcommand{\hlkwc}[1]{\textcolor[rgb]{0.333,0.667,0.333}{#1}}%
\newcommand{\hlkwd}[1]{\textcolor[rgb]{0.737,0.353,0.396}{\textbf{#1}}}%
\let\hlipl\hlkwb

\usepackage{ulem}

\usepackage{framed}
\makeatletter
\newenvironment{kframe}{%
 \def\at@end@of@kframe{}%
 \ifinner\ifhmode%
  \def\at@end@of@kframe{\end{minipage}}%
  \begin{minipage}{\columnwidth}%
 \fi\fi%
 \def\FrameCommand##1{\hskip\@totalleftmargin \hskip-\fboxsep
 \colorbox{shadecolor}{##1}\hskip-\fboxsep
     % There is no \\@totalrightmargin, so:
     \hskip-\linewidth \hskip-\@totalleftmargin \hskip\columnwidth}%
 \MakeFramed {\advance\hsize-\width
   \@totalleftmargin\z@ \linewidth\hsize
   \@setminipage}}%
 {\par\unskip\endMakeFramed%
 \at@end@of@kframe}
\makeatother

\definecolor{shadecolor}{rgb}{.97, .97, .97}
\definecolor{messagecolor}{rgb}{0, 0, 0}
\definecolor{warningcolor}{rgb}{1, 0, 1}
\definecolor{errorcolor}{rgb}{1, 0, 0}
\newenvironment{knitrout}{}{} % an empty environment to be redefined in TeX

\usepackage{alltt}
\usepackage{graphicx, fancyhdr}
\usepackage{amsmath, amsfonts}
\usepackage{color}
\usepackage{hyperref}

\newcommand{\blue}[1]{{\color{blue} #1}}

\setlength{\topmargin}{-.5 in} 
\setlength{\textheight}{9 in}
\setlength{\textwidth}{6.5 in} 
\setlength{\evensidemargin}{0 in}
\setlength{\oddsidemargin}{0 in} 
\setlength{\parindent}{0 in}
\newcommand{\ben}{\begin{enumerate}}
\newcommand{\een}{\end{enumerate}}


\lhead{STAT 305}
\chead{Homework 1} 
\rhead{Due Thursday, January 23rd in the class}
\lfoot{Spring 2020} 
\cfoot{\thepage} 
\rfoot{} 
\renewcommand{\headrulewidth}{0.4pt} 
\renewcommand{\footrulewidth}{0.4pt} 

\def\Exp#1#2{\ensuremath{#1\times 10^{#2}}}
\def\Case#1#2#3#4{\left\{ \begin{tabular}{cc} #1 & #2 \phantom
{\Big|} \\ #3 & #4 \phantom{\Big|} \end{tabular} \right.}
\IfFileExists{upquote.sty}{\usepackage{upquote}}{}
\begin{document}
\pagestyle{fancy} 

Show \textbf{all} of your work on this assignment and answer each question fully in the given context. 
Each individual part is worth 5 points and partial credit is awarded for close answers.
\vspace{0.3cm}

\textbf{If you cannot submit your homework in the class, you can drop it at my office door in 3220 Snedecore Hall by Thursday at 03:30 PM.}

\vspace{0.3cm}
\emph{Please} staple your assignment!

\begin{enumerate}

\item \textbf{[Ch. 1.2 Exercise 1, pg. 13]} Describe a situation in your field where an observational study might be used to answer a question of real importance. Describe another situation where an experiment might be used.[5pts]

\item \textbf{[Ch. 1.2 Exercise 2, pg. 13]} Describe two different contexts in your field where, respectively, qualitative and quantitative data might arise.[5pts] 

\item \textbf{ [Chapter 1, End of chapter execise, Exercise 8, pg. 24]} Consider a situation like that of Example 1 in the class (involving the heat treating of gears). Suppose that the original gears can be purchased from a variety of vendors, they can be made out of a variety of materials, they can be heated according to a variety of regimens (involving different times and temeratures), they can be cooled in a nuber of different ways, and the furnance atmosphere can be adjusted to a variety of different conditions. 

A number of features of the final gears are of interest, including their flatness, their concentricity, their hardness (both before and after heat treating), and their surface finish. 

    \begin{enumerate}
      \item What kind of data arise if, for a single set of conditions, the Rockwell hardness of several gears is measured both before and after heat treating? (Use terminology of section 1.2. i.e. what kind of data that would be in terms of univariate/multivariate, quantitative or qualitative, if quantitative, whether they are discerete or continuous, ...)[5 pts]
      \item In the same context, suppose that engineering specifications on flatness require that measured flatness not exceed .40 mm. If flatness is measured for several gears and each gear is simply marked Acceptable or Not Acceptable, what kind of data are generated? (i.e. what kind of data that would be in terms of univariate/multivariate, quantitative or qualitative, if quantitative, whether they are discerete or continuous, ...)[5 pts]
      \item Describe a three-factor full factorial study that might be carried out in this situation. Name the factors that will be used and describe the levels of each. Write out a list of all the different combinations of levels of the factors that will be studied. [10 pts]
    \end{enumerate}
    
\item \textbf{[Ch. 2.3 Exercise 1, pg. 47]} Consider the context of a study on making paper airplanes where two different Designs (say delta versus t wing), two different Papers (say construction versus typing) and two different Loading Conditions (with a paper clip versus without a paper clip) are of interest with regard to their impact on flight distance. Describe some variables that you would want to control in such a study. What are the response and experimental variables that would be appropriate in this context? Name a potential concomitant variable here.[5 pts]


\item \textbf{Hockey game attendance.}

Caroline performs the following study to see if outside temperature
has an effect on attendance at her college's hockey games. For each
hockey game at her college, Caroline records the outside temperature and the attendance. Here are her results:

\hspace{1in}
\begin{tabular}{|cr@{/}r|c|c|} \hline
\multicolumn{3}{|c|}{\emph{Day of Week}} & \emph{Temperature, deg. F} & \emph{Attendance} \\ \hline
Friday & 12&14 & 35 & 840 \\
Wednesday & 12&19 & 20 & 560 \\
Tuesday & 1&8 & $\!-5$ & 340 \\
Friday & 1&11 & 23 & 775 \\
Wednesday & 1&23 & 14 & 680 \\
Saturday & 2&2 & 30 & 950 \\
Friday & 2&8 & 28 & 950 \\
\hline
\end{tabular}
 
    \begin{enumerate} 
    \item Is this an experiment or observational study?[5 pts]
    
    \item What type of variable is attendance?[5 pts]
    
    \vspace{1cm}
    
    Caroline analyzes her results and finds that
    outside temperature and attendance have a strong
    positive correlation (i.e., as one increases, the other also increases).
    She concludes that higher game day temperatures causes higher
    attendance at their college's hockey games.
    

    \item Did she come to a proper conclusion for this study? Why or why not?[5 pts]
    
    \item Look at the day of the week of the hockey games. What type of variable is this?[5 pts]
     
    \item 
    Rewrite the data table, adding a new column ``School Night" (using the values ``no" if the game is on a Friday or Saturday, and ``yes" if the game is on any other day).
    How does Attendance relate to School Night?[5 pts]
    
   
    \end{enumerate}
    
\item \textbf{Washer stretching.}
  
George works for a company that manufactures  rubber washers.  He randomly selects 1000  washers off the assembly line throughout two weeks for a study on the durability of these washers under   stretching.  To make sure that the washers  are fit to be used in the real world, George must test  the washers.  Holding heat constant,  George subjects each washer to one of various methods of  stretching.  The washers are randomly assigned to be stretched under one of five different forces (low, medium-low, medium, medium-high, and high). After each test, George classifies  a washer as either defective or non-defective. 
 
    \begin{enumerate}
    \item Is this an experiment or observational study? [5 pts]
    
    \item What type of variable is heat? [5 pts]
     
    \item What type of variable is the amount of stretching?[5 pts] 
     
    \item What type of variable is response to the stretching method?[5 pts]
     
    \item The 1000 selected washers constitutes the sample. What is the population? [5 pts]
    
    \item George analyzes the results and finds that the defect rate increases with the amount of stretching. Can George conclude that the amount of stretching  causes a change in the defect rate of the washers? Why or why not?[5 pts]
    
    %Remember, though, that there always exists the chance that the sample
    %was not properly representative of the population, and perhaps 
    %the 100 washers he selected 
    %were some of the few washers in the population
    %that showed no effect. 
    %But through proper
    %experimental design, this chance is minimized and George can 
    %be highly confident in the results. More on confidence in Chapter 6.
    \end{enumerate}


\end{enumerate}

\emph{Total: 90 pts}
\end{document}
