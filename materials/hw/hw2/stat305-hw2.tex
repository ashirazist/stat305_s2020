\documentclass[11pt]{article}\usepackage[]{graphicx}\usepackage[]{color}
%% maxwidth is the original width if it is less than linewidth
%% otherwise use linewidth (to make sure the graphics do not exceed the margin)
\makeatletter
\def\maxwidth{ %
  \ifdim\Gin@nat@width>\linewidth
    \linewidth
  \else
    \Gin@nat@width
  \fi
}
\makeatother
\usepackage{pdfpages} 
\definecolor{fgcolor}{rgb}{0.345, 0.345, 0.345}
\newcommand{\hlnum}[1]{\textcolor[rgb]{0.686,0.059,0.569}{#1}}%
\newcommand{\hlstr}[1]{\textcolor[rgb]{0.192,0.494,0.8}{#1}}%
\newcommand{\hlcom}[1]{\textcolor[rgb]{0.678,0.584,0.686}{\textit{#1}}}%
\newcommand{\hlopt}[1]{\textcolor[rgb]{0,0,0}{#1}}%
\newcommand{\hlstd}[1]{\textcolor[rgb]{0.345,0.345,0.345}{#1}}%
\newcommand{\hlkwa}[1]{\textcolor[rgb]{0.161,0.373,0.58}{\textbf{#1}}}%
\newcommand{\hlkwb}[1]{\textcolor[rgb]{0.69,0.353,0.396}{#1}}%
\newcommand{\hlkwc}[1]{\textcolor[rgb]{0.333,0.667,0.333}{#1}}%
\newcommand{\hlkwd}[1]{\textcolor[rgb]{0.737,0.353,0.396}{\textbf{#1}}}%
\let\hlipl\hlkwb

\usepackage{ulem}

\usepackage{framed}
\makeatletter
\newenvironment{kframe}{%
 \def\at@end@of@kframe{}%
 \ifinner\ifhmode%
  \def\at@end@of@kframe{\end{minipage}}%
  \begin{minipage}{\columnwidth}%
 \fi\fi%
 \def\FrameCommand##1{\hskip\@totalleftmargin \hskip-\fboxsep
 \colorbox{shadecolor}{##1}\hskip-\fboxsep
     % There is no \\@totalrightmargin, so:
     \hskip-\linewidth \hskip-\@totalleftmargin \hskip\columnwidth}%
 \MakeFramed {\advance\hsize-\width
   \@totalleftmargin\z@ \linewidth\hsize
   \@setminipage}}%
 {\par\unskip\endMakeFramed%
 \at@end@of@kframe}
\makeatother

\definecolor{shadecolor}{rgb}{.97, .97, .97}
\definecolor{messagecolor}{rgb}{0, 0, 0}
\definecolor{warningcolor}{rgb}{1, 0, 1}
\definecolor{errorcolor}{rgb}{1, 0, 0}
\newenvironment{knitrout}{}{} % an empty environment to be redefined in TeX

\usepackage{alltt}
\usepackage{graphicx, fancyhdr}
\usepackage{amsmath, amsfonts}
\usepackage{color}
\usepackage{hyperref}

\newcommand{\blue}[1]{{\color{blue} #1}}

\setlength{\topmargin}{-.5 in} 
\setlength{\textheight}{9 in}
\setlength{\textwidth}{6.5 in} 
\setlength{\evensidemargin}{0 in}
\setlength{\oddsidemargin}{0 in} 
\setlength{\parindent}{0 in}
\newcommand{\ben}{\begin{enumerate}}
\newcommand{\een}{\end{enumerate}}


\lhead{STAT 305}
\chead{Homework \#2} 
\rhead{Due Thursday, Sep. $12^{th}$ in the class}
\lfoot{Fall 2019} 
\cfoot{\thepage} 
\rfoot{} 
\renewcommand{\headrulewidth}{0.4pt} 
\renewcommand{\footrulewidth}{0.4pt} 

\def\Exp#1#2{\ensuremath{#1\times 10^{#2}}}
\def\Case#1#2#3#4{\left\{ \begin{tabular}{cc} #1 & #2 \phantom
{\Big|} \\ #3 & #4 \phantom{\Big|} \end{tabular} \right.}
\IfFileExists{upquote.sty}{\usepackage{upquote}}{}
\begin{document}
\pagestyle{fancy} 

Show \textbf{all} of your work on this assignment and answer each question fully in the given context. \\
You will want to understand Exercise 1 from Section 2.1 before attempting the following questions.  
Your answers should be written in complete sentences.
It is possible that a drawing or table may help make your thoughts more concrete or illustrate a concept that would be difficult to describe in words alone - if so I encourage you to use one.\\

\emph{Please} staple your assignment!

\begin{itemize}

\item \textbf{Problem1: Chapter 2, Section 3, Exercise 1 (page 47)[5 pts]} \\

\item \textbf{Problem2: Chapter 2, Section 3, Exercise 5 (page 47)[5 pts]}\\


\item \textbf{Problem3: Chapter 2, End of chapter exercise, Exercise 7 (page 65)[10 pts]}\\

\item \textbf{Problem4: Chapter 2, End of chapter exercise, Exercise 11 (page 65)[5 pts each part]}\\
\item \textbf{Problem5: Chapter 3, Section 2,  Exercise 1 (page 92)[5 pts each part]}\\

\item \textbf{Problem6: JMP Assignment. [5 pts each part]} 

   Computing is one of the most important parts of modern data analysis. A large part of data science simply wouldn't exist without the tools developed by scientists working at the intersections of computer science, mathematics, and statistics. 
   Because of that, there will inevitably be parts of this course where a statistical computing tools are needed. SAS and R are the two main languages used by statisticians, though Python, Julia, F\#, C++ and others are also common.
   SAS has a software called JMP ("Jump") that makes doing statistical analyses simpler - it is more powerful than Excel or your calculator but requires little in the sense of coding making the learning curve much lower. 
   We will be using it this semester. There are labs in Snedecor Hall with the software pre-installed, but it is free for students and I encourage you to download a copy for yourself using the link below.

   Download: \href{https://www.stat.iastate.edu/statistical-software}{https://www.stat.iastate.edu/statistical-software}

   In this problem, we will work through the tutorial found at \href{http://web.utk.edu/~cwiek/201Tutorials/}{http://web.utk.edu/$\sim$cwiek/201Tutorials/}. Download and install \texttt{JMP} or find a computer with it already installed. Once you have done this, complete the following sections from the the tutorial linked above. For each part, print and include the table or graph produced. (Note: You can save the plots/tables and combine them in a single document).\\
\textbf{NOTE:} You should use the data set available on the tutorial to finish this exercise, but if you have any data sets of your own, I'd be more than happy to see you use JMP to make some summarizations on your data set. 

   \begin{enumerate}
      \item Creating a JMP data table
      \item Bar Chart
      \item Pie Chart
      \item Mosaic Plot
      \item Histogram and Box Plot
      \item Stem and Leaf Plot
      \item Side-by-Side Box Plots
      \item Calculating Summary Statistics of Quantitative Data
   \end{enumerate}

Total: 90 pts


















\end{itemize}


\end{document}
