\documentclass[11pt]{article}\usepackage[]{graphicx}\usepackage[]{color}
%% maxwidth is the original width if it is less than linewidth
%% otherwise use linewidth (to make sure the graphics do not exceed the margin)
\makeatletter
\def\maxwidth{ %
  \ifdim\Gin@nat@width>\linewidth
    \linewidth
  \else
    \Gin@nat@width
  \fi
}
\makeatother

\definecolor{fgcolor}{rgb}{0.345, 0.345, 0.345}
\newcommand{\hlnum}[1]{\textcolor[rgb]{0.686,0.059,0.569}{#1}}%
\newcommand{\hlstr}[1]{\textcolor[rgb]{0.192,0.494,0.8}{#1}}%
\newcommand{\hlcom}[1]{\textcolor[rgb]{0.678,0.584,0.686}{\textit{#1}}}%
\newcommand{\hlopt}[1]{\textcolor[rgb]{0,0,0}{#1}}%
\newcommand{\hlstd}[1]{\textcolor[rgb]{0.345,0.345,0.345}{#1}}%
\newcommand{\hlkwa}[1]{\textcolor[rgb]{0.161,0.373,0.58}{\textbf{#1}}}%
\newcommand{\hlkwb}[1]{\textcolor[rgb]{0.69,0.353,0.396}{#1}}%
\newcommand{\hlkwc}[1]{\textcolor[rgb]{0.333,0.667,0.333}{#1}}%
\newcommand{\hlkwd}[1]{\textcolor[rgb]{0.737,0.353,0.396}{\textbf{#1}}}%
\let\hlipl\hlkwb

\usepackage{ulem}

\usepackage{framed}
\makeatletter
\newenvironment{kframe}{%
 \def\at@end@of@kframe{}%
 \ifinner\ifhmode%
  \def\at@end@of@kframe{\end{minipage}}%
  \begin{minipage}{\columnwidth}%
 \fi\fi%
 \def\FrameCommand##1{\hskip\@totalleftmargin \hskip-\fboxsep
 \colorbox{shadecolor}{##1}\hskip-\fboxsep
     % There is no \\@totalrightmargin, so:
     \hskip-\linewidth \hskip-\@totalleftmargin \hskip\columnwidth}%
 \MakeFramed {\advance\hsize-\width
   \@totalleftmargin\z@ \linewidth\hsize
   \@setminipage}}%
 {\par\unskip\endMakeFramed%
 \at@end@of@kframe}
\makeatother

\definecolor{shadecolor}{rgb}{.97, .97, .97}
\definecolor{messagecolor}{rgb}{0, 0, 0}
\definecolor{warningcolor}{rgb}{1, 0, 1}
\definecolor{errorcolor}{rgb}{1, 0, 0}
\newenvironment{knitrout}{}{} % an empty environment to be redefined in TeX

\usepackage{xcolor}
\usepackage{alltt}
\usepackage{graphicx, fancyhdr}
\usepackage{amsmath, amsfonts}
\usepackage{color}
\usepackage{hyperref}

\newcommand{\blue}[1]{{\color{blue} #1}}

\setlength{\topmargin}{-.5 in} 
\setlength{\textheight}{9 in}
\setlength{\textwidth}{6.5 in} 
\setlength{\evensidemargin}{0 in}
\setlength{\oddsidemargin}{0 in} 
\setlength{\parindent}{0 in}
\newcommand{\ben}{\begin{enumerate}}
\newcommand{\een}{\end{enumerate}}


\lhead{STAT 305}
\chead{Homework 6} 
\rhead{Due Thursday, March $12^{th}$ in the class}
\lfoot{Spring 2020} 
\cfoot{\thepage} 
 
\renewcommand{\headrulewidth}{0.4pt} 
\renewcommand{\footrulewidth}{0.4pt} 

\def\Exp#1#2{\ensuremath{#1\times 10^{#2}}}
\def\Case#1#2#3#4{\left\{ \begin{tabular}{cc} #1 & #2 \phantom
{\Big|} \\ #3 & #4 \phantom{\Big|} \end{tabular} \right.}
\IfFileExists{upquote.sty}{\usepackage{upquote}}{}
\usepackage{Sweave}
\begin{document}
\Sconcordance{concordance:stat305_hw6.tex:stat305_hw6.Rnw:%
1 83 1 1 0 85 1}

\pagestyle{fancy} 

Show \textbf{all} of your work on this assignment and answer each question fully in the given context. 

\vspace{0.3cm}

\textbf{If you cannot submit your homework in the class, you can drop it at my office door in 3220 Snedecore Hall by Thursday at 03:30 PM.}

\vspace{0.3cm}

\emph{Please} staple your assignment and write your name !

\begin{enumerate}
	
  \item \textbf{[Ch. 5.1, Exercise 9, pg. 244]} Transmission line interruptions in a telecommunication network occur at an average rate of one per day.
    \begin{enumerate}
      \item Use a Poisson distribution as a model for
        $$
        X = \text{the number of interruptions in the next five-day work week}
        $$
      Now, precisely specify the probability distribution.[5pts]
      \item Find $P[X = 0]$[5pts]
      \item Now consider the random variable
        $$
        Y = \text{the number of work weeks in the next four in which there are no interruptions}
        $$
        What is a reasonable probability model for $Y$?[5pts]\\
        \emph{hint:}i.e. precisely specify the probability distribution.
        
      \item Find $P[Y = 2]$.[5pts]
	\end{enumerate}
	\item Suppose a standup comedian plans to give a total of $n=5$ jokes in an entire 2-hour performance. Call a joke a success if at least one audience member laughs. If no audience member laughs, the joke is a failure.\\
	Assume that all the jokes are equally funny, with $p= P(\text{success}) = 0.2$. Let $X$ be the random variable associated with the number of successful jokes out of the total 5. 
	\begin{enumerate}
	  \item Preciisely state the distribution of $X$, giving the values of any parameters necessary.[2pts]
	  \item Calculate the probability that the whole night is a failure. i.e. find the $P(\text{no success})$. [5pts]
	  \item Calculate the probability that the comedian tells at least $4$ successful jokes.[5pts]
	  \item Calculate the expected number of successful jokes.[5pts]
	  \item Calculate the \underline{standard deviation} of the successful jokes.[5pts]
	  
	\end{enumerate}
	
	\item  The  number  of  computer  shutdowns  during  any  month  has  a  Poisson  distribution,  averaging $0.25$ shutdowns per month.
	\begin{enumerate}
	  \item What is the probability of at least $2$ computer shutdowns during the next \emph{year}?[5pts]
	  \item What is the probability of at most $2$ computer shutdowns during the next \emph{6 month}? [5pts]
	  \item What is the variance of the number of computer shutdowns during the next \emph{year}? [2pts]
	\end{enumerate}
	
	
	\item Suppose that $X$ is a random variable with probability density function $$ f(x) = \begin{cases} c \cdot x^2 & - 2 \le x \le 2 \\ 0 & o.w. \end{cases} $$
 
        \begin{enumerate}
               \item Find the value of $c$ that makes $f(x)$ a valid probability density function.[5 pts]
               
                \item Find the CDF of the random variable $X$.[5 pts]
                
                \item What is $P(\vert{X}\vert \ge {-1})$ [5 pts].
                
                \item Find the expected value of $X$.[ 5 pts]
                
        \end{enumerate}      
    	
	\item  Consider a continuously distributed random variable, $W$, with a probability density function given by 
	$$ f(w) = \begin{cases} \frac{1}{5(1 - e^{-2})} e^{-w/5} & 0 \le w \le 10 \\ 0 & \text{otherwise} \end{cases} $$
	
	    \begin{enumerate}
                  \item  Show that the function $f(w)$ is a valid probability density function (i.e., show that (i) $f(w)$ is non-negative and (ii) $\int_{-\infty}^{\infty} f(w) dw = 1$). [5 pts]
                   \item Find $P(W \le 2)$ [5 pts]
                   
                   \item  Find $P(2 \le W \le 10)$ [5 pts]
                   
                   \item  Find $E(W)$ [5 pts]

      \end{enumerate}

  \item \textbf{[Ch. 5.2, Exercise 1, pg. 263]} The random number generator supplied on a calculator is not terribly well chosen, in that values it generates are not adequately described by a uniform distribution on the interval $(0,1)$. Suppose instead that a probability density
    $$
    f(x) = \begin{cases} k(5 - x) & 0 < x< 1 \\ 0 & \text{otherwise}\end{cases}
    $$
    
    is a more appropriate model for $X =$ the next value produced by this random number generator.
    \begin{enumerate}
      \item Find the value of $k$.[5pts]
      \item Evaluate $P[.25 < X < .75]$[5pts]
      \item Compute the cumulative probability distribution function for $X$, $F(x)$. [5pts]
      \item Calculate $\text{E}(X)$ and the standard deviation of $X$.[6pts]
    \end{enumerate}
\end{enumerate}
[Total: 115 pts]



\end{document}
