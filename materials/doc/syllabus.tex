% For LaTeX-Box: root = syllabus_stat105_F15.tex 
%%%%%%%%%%%%%%%%%%%%%%%%%%%%%%%%%%%%%%%%%%%%%%%%%%%%%%%%%%%%%%%%%%%%%%%%%%%%%%%%
%  File Name: syllabus_stat105_F15.tex
%  Purpose:
%
%  Creation Date: 24-08-2015
%  Last Modified: Tue Aug 22 03:31:41 2017
%  Created By:
%%%%%%%%%%%%%%%%%%%%%%%%%%%%%%%%%%%%%%%%%%%%%%%%%%%%%%%%%%%%%%%%%%%%%%%%%%%%%%%%
\documentclass[12pt]{article}\usepackage[]{graphicx}\usepackage[]{color}
%% maxwidth is the original width if it is less than linewidth
%% otherwise use linewidth (to make sure the graphics do not exceed the margin)
\makeatletter
\def\maxwidth{ %
  \ifdim\Gin@nat@width>\linewidth
    \linewidth
  \else
    \Gin@nat@width
  \fi
}
\makeatother

\definecolor{fgcolor}{rgb}{0.345, 0.345, 0.345}
\newcommand{\hlnum}[1]{\textcolor[rgb]{0.686,0.059,0.569}{#1}}%
\newcommand{\hlstr}[1]{\textcolor[rgb]{0.192,0.494,0.8}{#1}}%
\newcommand{\hlcom}[1]{\textcolor[rgb]{0.678,0.584,0.686}{\textit{#1}}}%
\newcommand{\hlopt}[1]{\textcolor[rgb]{0,0,0}{#1}}%
\newcommand{\hlstd}[1]{\textcolor[rgb]{0.345,0.345,0.345}{#1}}%
\newcommand{\hlkwa}[1]{\textcolor[rgb]{0.161,0.373,0.58}{\textbf{#1}}}%
\newcommand{\hlkwb}[1]{\textcolor[rgb]{0.69,0.353,0.396}{#1}}%
\newcommand{\hlkwc}[1]{\textcolor[rgb]{0.333,0.667,0.333}{#1}}%
\newcommand{\hlkwd}[1]{\textcolor[rgb]{0.737,0.353,0.396}{\textbf{#1}}}%

\usepackage{framed}
\makeatletter
\newenvironment{kframe}{%
 \def\at@end@of@kframe{}%
 \ifinner\ifhmode%
  \def\at@end@of@kframe{\end{minipage}}%
  \begin{minipage}{\columnwidth}%
 \fi\fi%
 \def\FrameCommand##1{\hskip\@totalleftmargin \hskip-\fboxsep
 \colorbox{shadecolor}{##1}\hskip-\fboxsep
     % There is no \\@totalrightmargin, so:
     \hskip-\linewidth \hskip-\@totalleftmargin \hskip\columnwidth}%
 \MakeFramed {\advance\hsize-\width
   \@totalleftmargin\z@ \linewidth\hsize
   \@setminipage}}%
 {\par\unskip\endMakeFramed%
 \at@end@of@kframe}
\makeatother

\definecolor{shadecolor}{rgb}{.97, .97, .97}
\definecolor{messagecolor}{rgb}{0, 0, 0}
\definecolor{warningcolor}{rgb}{1, 0, 1}
\definecolor{errorcolor}{rgb}{1, 0, 0}
\newenvironment{knitrout}{}{} % an empty environment to be redefined in TeX

\usepackage{alltt}
\textwidth=7in
\textheight=9.5in
\topmargin=-1in
\headheight=0in
\headsep=.5in
\hoffset  -.85in

\pagestyle{empty}

%\renewcommand{\thefootnote}{\fnsymbol{footnote}}
\IfFileExists{upquote.sty}{\usepackage{upquote}}{}
\begin{document}

\begin{center}
   {\bf Statistics 305 \\ Engineering Statistics- Section 3}
\end{center}
\centerline{\textbf{Tue-Thu 09:30 - 10:50 AM, PEARSON 2115}}

\setlength{\unitlength}{1in}

\begin{picture}(6,.1) 
\put(0,0) {\line(1,0){6.25}}         
\end{picture}

\renewcommand{\arraystretch}{2}
 
\vskip.25in

\noindent\textbf{Course Description:}
\textit{(Prereq: MATH 165)}
Statistics for engineering problem solving. Principles of engineering data collection; descriptive statistics; elementary probability distributions; principles of experimentation; confidence intervals and significance tests; one-, two-, and multi-sample studies; regression analysis; use of statistical software.
\vskip.25in

\noindent\textbf{Learning Outcome:}
By the end of this course, students should learn basic concepts of statistics and probability to solve problems arising in engineering applications.
\vskip.25in

\noindent\textbf{Required Text:}  
\textit{Basic Engineering Data Collection and Analysis} by Stephen B. Vardeman and J. Marcus Jobe (ISBN 0-534-36957-X). 
\vskip.25in

\noindent\textbf{Instructor:} Amin Shirazi (ashirazi@iastate.edu, 3220 Snedecor Hall, 1-515-294-7891). 
\vskip.25in

\noindent\textbf{Resources}:
\begin{center} \begin{minipage}{6.5in}
\begin{flushleft}
Office hours \dotfill Tue 11:00-12:00, Fri 02:00-03:00 PM in 3220 Snedecor Hall. \\
TA \dotfill Li, Yusi (yusi1409@iastate.edu, office hours: Mon 12:00- 01:00, Wed  09:00- 10:00, 2404 Snedecore Hall) \\
Course page \dotfill https://ashirazist.github.io/stat305.github.io/index.html \\
Canvas\dotfill For grades (https://canvas.iastate.edu)\\
Software \dotfill JMP (free download at https://www.stat.iastate.edu/statistical-software)
\end{flushleft}
\end{minipage}
\end{center}
\vskip.25in

\noindent\textbf{Class attendance: }
You are responsible for all material presented in lecture and assigned as required reading. No Class during Nov 25-29 (Fall break)
\vskip.25in

\noindent\textbf{Important Dates}:
\begin{center} \begin{minipage}{6.5in}
\begin{flushleft}
Quiz 1  \dotfill September 26, Thursday \\
Quiz 2  \dotfill October 17, Thursday \\
Quiz 3  \dotfill November 14, Thursday \\
Fall break \dotfill Monday - Friday, November 25-29\\
Quiz 4 \dotfill December 5, Thursday \\
Course Final \dotfill December 18, Wednesday 09:45- 11:45., location TBA\\
\end{flushleft}
\end{minipage}
\end{center}
\vskip.25in

\noindent\textbf{Assessment Policy:} Grades (include plus/minus) will be determined based on the following:
\begin{center} 
\begin{minipage}{6.5in}
\begin{flushleft}
\textbf{Homework:} 
Most weeks, homework will be assigned to be collected the following week in class. To accommodate unexpected events that may impact students ability to complete this assignments on time I will drop the lowest of the homework grades from the overall average.\\
\end{flushleft}
\end{minipage}
\end{center}

\begin{center} 
\begin{minipage}{6.5in}
\begin{flushleft}
\textbf{Quizzes:} 
There will be four semester quizzes and a comprehensive final. The semester quizzes will be given during the lecture period and will be closed book. The final exam is comprehensive, and will be on December 18, Wednesday 09:45- 11:45. The location will be announced.
\end{flushleft}
\end{minipage}
\end{center}



\begin{center} 
\begin{minipage}{6.5in}
\begin{flushleft}
\textbf{Weight:} The components of a student's grade have the following weights:\\
\begin{center} 
\begin{minipage}{6in}
\begin{flushleft}
Homework \dotfill 12\%  (assigned weekly) \\
Quizzes \dotfill 48\%  (4 Quizzes, 12\% each) \\
Final Exam \dotfill 40\% (December 18, Wednesday 09:45- 11:45., location TBA) \\
\end{flushleft}
\end{minipage}
\end{center}
\end{flushleft}
\end{minipage}
\end{center}

\begin{center} 
\begin{minipage}{6.5in}
\begin{flushleft}
\textbf{Letter grades:} 
Letter grades will be assigned based on the following ranges: A = 100-93, A- = 90-93, B+ = 87-90, B = 83-87, B- = 80-83, C+ = 77-80, C = 73-77, C- = 70-73, D+ = 67-70, D = 63-67, D- = 60-63, F = 60-0
\end{flushleft}
\end{minipage}
\end{center}

\noindent\textbf{Course Outline:} 
\begin{center} \begin{minipage}{6.5in}
\begin{flushleft}
Introduction, Data Collection (Ch.1, 2) \dotfill Read syllabus carefully, HW1 assigned\\
Descriptive Statistics (sec. 3.1 3.2)  \dotfill HW1 due, HW2 assigned\\
Descriptive Statistics, Line Fitting  (Ch. 3.3, 4.1, 4.2) \dotfill HW2 due, HW3 assigned\\
Curve and Surface Fitting (sec. 4.2) \dotfill JMP, HW3 due, HW4 assigned \\
Random Variables (sec. 5.1) \dotfill \textbf{Quiz 1}, No homework assigned\\
Random Variables (sec. 5.1) \dotfill HW4 due, HW5 assigned, covering Ch. 1 - 4 \\
Random Variables (sec. 5.2) \dotfill HW5 due, HW6 assigned  \\
Random Variables (sec. 5.2) \dotfill HW6 due, HW7 assigned  \\
Random Variables (sec. 5.4, 5.5) \dotfill \textbf{Quiz 2}, No homework assigned\\
Random Variables, Simple Inference (sec. 5.5, 6.1) \dotfill HW7 due, HW8 assigned \\
Simple Inference (6.1, 6.2) \dotfill HW8 due, HW9 assigned  \\
Simple Inference (6.2, 6.3) \dotfill \textbf{Quiz 3}, covering Ch. 5 \\
Simple Inference (6.2, 6.3) \dotfill HW9 due, HW10 assigned  \\
Simple Inference (6.3, 6.6) \dotfill HW10 due, No homework assigned\\
Fall  break \dotfill No homework assigned \\
One Way Model CI's for Linear Functions of Means (7.1, 7.2) \dotfill \textbf{Quiz 4} HW11 assigned  \\
One Way ANOVA, Control Charts, Review (7.4, 7.5) \dotfill HW11 due \\
\textbf{Final Exam} \dotfill  December 18, Wednesday 09:45- 11:45., location TBA.
\end{flushleft}
\end{minipage}
\end{center}

\noindent\textbf{Attendance}:  
Grades do not directly depend on attendance - still, experience shows that attendance and course performance are significantly related to each other. In order to get the most out of this course and do his or her personal best, it is necessary for a student to treat attendance as if it were mandatory.

\vskip.25in
\noindent\textbf{Academic Honesty}:  
As an Iowa State University student, you have agreed to abide by the University's academic honesty policy (http://www.dso.iastate.edu/ja/academic/misconduct.html).
Academic misconduct is a serious matter and student's suspected of academic dishonesty will be reported to the Dean of Students Office.
Questions related to course assignments and the academic honesty policy should be directed to the instructor.

\vskip.25in
\noindent\textbf{Extra Help}:  
Do not hesitate to come to my office during office hours or schedule an appointment to discuss a homework problem or any aspect of the course. If you want to hire an outsider tutor (i.e., for a fee), you can find possible tutors through the statistics department.

\vskip.25in
\noindent\textbf{Accessibility Statement}: 

Iowa State University is committed to assuring that all educational activities are free from discrimination and harassment based on disability status.  Students requesting accommodations for a documented disability are required to work directly with staff in Student Accessibility Services (SAS) to establish eligibility and learn about related processes before accommodations will be identified.  After eligibility is established, SAS staff will create and issue a Notification Letter for each course listing approved reasonable accommodations.  This document will be made available to the student and instructor either electronically or in hard-copy every semester.  Students and instructors are encouraged to review contents of the Notification Letters as early in the semester as possible to identify a specific, timely plan to deliver/receive the indicated accommodations. Reasonable accommodations are not retroactive in nature and are not intended to be an unfair advantage.  Additional information or assistance is available online at www.sas.dso.iastate.edu, by contacting SAS staff by email at accessibility@iastate.edu, or by calling 515-294-7220. Student Accessibility Services is a unit in the Dean of Students Office located at 1076 Student Services Building.

\vskip.25in
\noindent\textbf{Disability Accommodation}:  
Iowa State University complies with the Americans with Disabilities Act and Sect 504 of the Rehabilitation Act. If you have a disability and anticipate needing accommodations in this course, please contact your instructor (in this case, Ian Mouzon) to set up a meeting within the first two weeks of the semester or as soon as you become aware of your need. Before meeting with the instructor, you will need to obtain a SAAR form with recommendations for accommodations from the Disability Resources Office, located in Room 1076 on the main floor of the Student Services Building. Their telephone number is 515-294-7220 or email disabilityresources@iastate.edu. Retroactive requests for accommodations will not be honored.

\vskip.25in
\noindent\textbf{Dead Week}:
This class follows the Iowa State University Dead Week policy as noted in section 10.6.4 of the Faculty Handbook http://www.provost.iastate.edu/resources/faculty-handbook .

\vskip.25in
\noindent\textbf{Harassment and Discrimination}:
Iowa State University strives to maintain our campus as a place of work and study for faculty, staff, and students that is free of all forms of prohibited discrimination and harassment based upon race, ethnicity, sex (including sexual assault), pregnancy, color, religion, national origin, physical or mental disability, age, marital status, sexual orientation, gender identity, genetic information, or status as a U.S. veteran. Any student who has concerns about such behavior should contact his/her instructor, Student Assistance at 515-294-1020 or email dso-sas@iastate.edu, or the Office of Equal Opportunity and Compliance at 515-294-7612.

\vskip.25in
\noindent\textbf{Religious Accommodation}:
If an academic or work requirement conflicts with your religious practices and/or observances, you may request reasonable accommodations. Your request must be in writing, and your instructor or supervisor will review the request.  You or your instructor may also seek assistance from the Dean of Students Office or the Office of Equal Opportunity and Compliance.

\vskip.25in
\noindent\textbf{Contact Information}:
If you are experiencing, or have experienced, a problem with any of the above issues, email academicissues@iastate.edu.

\end{document}
